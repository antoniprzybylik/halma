\documentclass[12pt, titlepage]{article}
\usepackage[polish]{babel}
\usepackage[utf8]{inputenc}
\usepackage[T1]{fontenc}
\usepackage{nunito}
\usepackage[margin=1.2in]{geometry}
\usepackage{ragged2e}
\usepackage{graphicx}
\usepackage{grffile}
\usepackage{longtable}
\usepackage{wrapfig}
\usepackage{rotating}
\usepackage[normalem]{ulem}
\usepackage{amsmath}
\usepackage{textcomp}
\usepackage{amssymb}
\usepackage{capt-of}
\usepackage{hyperref}
\author{Antoni Przybylik\\ PIPR 22Z\\ GRUPA 101}
\date{\today}
\title{HALMA - Dokumentacja Projektu}

\setcounter{secnumdepth}{0}
\begin{document}
\maketitle
\justifying

\section{Zadanie}
Napisać program grający w Halma. Powinna być możliwość gry:

\begin{itemize}
\item dwóch osób ze sobą,
\item osoby z komputerem,
\item komputera z komputerem.
\end{itemize}

\noindent
Program powinien kontrolować poprawność wykonywanych ruchów. Interfejs z użytkownikiem może być tekstowy.
\\~\\
Bardzo istotną częścią zadania jest opracowanie i zaimplementowanie jak najlepszego algorytmu grającego w grę.
\\~\\
Moduł realizujący algorytm gry komputera musi być wydzielony.

\section{Cel i opis projektu}
Lorem ipsum dolor sit amet, consectetur adipiscing elit, sed do eiusmod tempor incididunt ut labore et dolore magna aliqua. Ut enim ad minim veniam, quis nostrud exercitation ullamco laboris nisi ut aliquip ex ea commodo consequat. Duis aute irure dolor in reprehenderit in voluptate velit esse cillum dolore eu fugiat nulla pariatur. Excepteur sint occaecat cupidatat non proident, sunt in culpa qui officia deserunt mollit anim id est laborum.

\section{Architektura}
Ante metus dictum at tempor. Placerat orci nulla pellentesque dignissim enim sit. Integer enim neque volutpat ac tincidunt vitae semper quis. Massa tempor nec feugiat nisl pretium fusce id velit ut. At elementum eu facilisis sed. Ac orci phasellus egestas tellus rutrum tellus. Nunc scelerisque viverra mauris in aliquam sem. Risus nullam eget felis eget nunc lobortis mattis aliquam. Sed euismod nisi porta lorem mollis. Amet commodo nulla facilisi nullam vehicula. Mauris nunc congue nisi vitae. Sit amet nisl purus in mollis nunc sed. Pellentesque pulvinar pellentesque habitant morbi. Ultricies integer quis auctor elit sed. Lobortis scelerisque fermentum dui faucibus in ornare quam. Tristique nulla aliquet enim tortor at auctor urna nunc.

\section{Wymagania}
Żeby móc uruchomić grę w trbie TUI należy mieć zainstalowaną
bibliotekę curses.
\\~\\
Interfejs TUI działa tylko w terminalach obsługujących 8-bitowe
kolory i pozwalających na zmianę ich wartości\footnote{Każdy nowoczesny terminal jak urxvt, xfce4-terminal, gnome-terminal powinien spełniać te wymagania. Gra nie działa w xtermie.}.
Dodatkowo, jest wymagane żeby okno terminala miało rozmiar co
najmniej 40x70\footnote{Wysokość x Szerokość.}.

\section{Instrukcja użycia}
W ramach projektu jest zaimplementowany interfejs tekstowy TUI.


\section{Część refleksyjna}
Dolor sed viverra ipsum nunc. Egestas sed tempus urna et pharetra pharetra massa massa. Tempus iaculis urna id volutpat lacus laoreet. Augue neque gravida in fermentum et sollicitudin ac orci phasellus. Malesuada bibendum arcu vitae elementum curabitur vitae nunc sed. Enim blandit volutpat maecenas volutpat blandit aliquam. Ut sem nulla pharetra diam sit amet nisl suscipit adipiscing. Vestibulum rhoncus est pellentesque elit ullamcorper dignissim cras tincidunt. Eget nunc lobortis mattis aliquam. Convallis tellus id interdum velit.

\end{document}
