\documentclass[12pt, titlepage]{article}
\usepackage[polish]{babel}
\usepackage[utf8]{inputenc}
\usepackage[T1]{fontenc}
\usepackage{nunito}
\usepackage[margin=1in]{geometry}
\usepackage{ragged2e}
\usepackage{graphicx}
\usepackage{grffile}
\usepackage{longtable}
\usepackage{wrapfig}
\usepackage{rotating}
\usepackage[normalem]{ulem}
\usepackage{amsmath}
\usepackage{textcomp}
\usepackage{amssymb}
\usepackage{capt-of}
\usepackage{hyperref}
\author{Antoni Przybylik\\ PIPR 22Z\\ GRUPA 101}
\date{\today}
\title{HALMA - Dokumentacja Projektu}

\setcounter{secnumdepth}{0}
\begin{document}
\maketitle
\justifying

\section{Zadanie}
Napisać program grający w Halma. Powinna być możliwość gry:

\begin{itemize}
\item dwóch osób ze sobą,
\item osoby z komputerem,
\item komputera z komputerem.
\end{itemize}

\noindent
Program powinien kontrolować poprawność wykonywanych ruchów. Interfejs z użytkownikiem może być tekstowy.
\\~\\
Bardzo istotną częścią zadania jest opracowanie i zaimplementowanie jak najlepszego algorytmu grającego w grę.
\\~\\
Moduł realizujący algorytm gry komputera musi być wydzielony.

\section{Cel i opis projektu}
Celem projektu było zaprojektowanie i napisanie gry Halma w
języku Python,
przygotowanie testów jednostkowych i dokumentacji.
W tym celu należało wykorzystać umiejętności zdobyte
na przedmiocie "`Podstawy Informatyki i Programowania"'.

\pagebreak
\section{Architektura}
Projekt jest podzielone na trzy główne moduły:
\begin{itemize}
	\item Silnik gry (klasa Engine).
	\item Interfejsy użytkownika.
	\item Klasy graczy (klasy pochodne od Player).
\end{itemize}

\noindent
W silniku zaimplementowano bazową funkcjonalność gry.
UI odpowiada za komunikację z użytkownikiem. W ramach
projektu zaimplementowałem tylko jedno UI (w formie TUI),
ale aplikacja
jest tak zaprojektowana, że istnieje możliwość stworzenia
wielu UI, które będą komunikować się z resztą aplikacji
w taki sam sposób jak HalmaTui (klasa implementująca interfejs TUI).
Klasy graczy, natomiast, odpowiadają za wykonywanie ruchów,
mogą to być boty lub ludzie. Kiedy użytkownik zarząda wykonania
ruchu wciskając odpowiedni przycisk w UI wywoływana jest metoda
make\_move() dla obecnie ruszającego się gracza. Jeśli ten gracz
jest botem - po prostu wykonuje ruch. Gracz-Człowiek ma możliwość
wyświetlenia okna dialogowego z zapytaniem o ruch.
\\~\\
W aplikacji wydzieliłem też trzy moduły pomocnicze:
\begin{itemize}
\item Silnik UI.
\item Interfejs silnika gry dla użytkownika.
\item Klasę Game reprezentującą stan całej gry.
\end{itemize}

\noindent
Oddzielenie Silnika UI (w klasie TuiEngine) od
modułu wyświetlającego UI ma przyczynę w tym, że
elementy UI mają mieć możliwość wyświetlać także
niektóre obiekty klas pochodnych od Player (gracze-ludzie).
\\~\\
Interfejs silnika gry dla użytkownika (w klasie GameInterface)
ma za zadanie tworzyć warstwę abstrakcji pomiędzy UI, a silnikiem
gry. Tu przekazywany jest napis, który wpisał użytkownik w okienku
dialogowym i rozbijany jest on na konkretne współrzędne i
wykonywany jest ruch wykorzystując niżej-poziomowy interfejs
silnika gry (w klasie Engine).

\pagebreak
\noindent
\textbf{Klasa Engine} implementuje bazową funkcjonalność
gry. Znajdują tam się między innymi metody
odpowedzialne za ustawianie/odczyt pól na planszy,
metody pozwalające na sprawdzenie możliwych ruchów
z danego pola. W tej klasie jest zapisany stan planszy
i numer obecnego ruchu.
\\~\\
\textbf{Klasa HalmaTui} implementuje tekstowy interfejs
użytkownika (TUI) dla gry Halma. Używa on ogólnych
elementów interfejsu takich jak okna dialogowe, które
są dostarczane przez silnik TUI (w klasie TuiEngine).
\\~\\
\textbf{Klasy pochodne od Player} implementują metodę
make\_move(), która odpowiada za wykonywanie ruchu. W
przypadku botów ruch jest po prostu wykonywany. Gracze-ludzie
wyświetlają okno dialogowe
przy użyciu interfejsu klasu TuiEngine (tej samej której używa
HalmaTui) pytającego gracza o ruch, a potem go wykonują.
\\~\\
\textbf{Klasa GameBot} dziedzicząca po Player jest klasą
bazową dla wszystkich botów (RandomBot, ForwardBot). Zawiera
metody wspólne dla wszystkich botów. W klasach pochodnych od
GameBot są zaimplementowane konkretne algorytmy dla każdego bota.
\\~\\
\textbf{Klasa TuiPlayer} dziedzicząca po Player implementuje
wykonywanie ruchu przez człowieka. Wyświetla okno dialogowe
z zapyteniem o ruch, wprowadzony ruch przekazuje do modułu
GameInterface.
\\~\\
\textbf{Klasa TuiEngine} zawiera metody
takie jak: wyświetlająca okno dialogowe, wyświetlająca splash screen.
Jest wykorzystywana przez HalmaTui do narysowania okna gry i
przez TuiPlayer do zapytań o ruch.
\\~\\
\textbf{Klasa GameInterface} wprowadza warstwę
abstrakcji pomiędzy UI, a silnik gry. Na przykład ruchy wpisane
przez użytkownika nie są sprawdzane i przetwarzane na współrzędne
w tych samych funkcjach które wyświetlają UI tylko są przekazywane
klasie GameInterface, która ma referencję na obiekt klasy Engine
i wykonuje
ona ruch. Referencję na obiekt klasy GameInterface ma UI i TuiPlayer,
który jest obiektem klasy Player implementującym grę przez użytkownika.
\\~\\
\textbf{Klasa Game} posiada referencję na obiekty klas: GameInterface, Engine i
obiekty reprezentujące gracza białego i czarnego (obiekty klas
pochodnych od Player). Ten stan można zapisać do pliku i wczytać
go z pliku.

\pagebreak
\section{Wymagania}
Żeby móc uruchomić grę w trbie TUI należy mieć zainstalowaną
bibliotekę curses i moduł timeout\_decorator.
\\~\\
Interfejs TUI działa tylko w terminalach obsługujących 8-bitowe
kolory i pozwalających na zmianę ich wartości\footnote{Każdy nowoczesny terminal jak urxvt, xfce4-terminal, gnome-terminal powinien spełniać te wymagania. Gra nie działa w xtermie.}.
Dodatkowo, terminal musi obsługiwać znaki UTF-8.
Jest wymagane żeby okno terminala miało rozmiar co
najmniej 40x70\footnote{Wysokość x Szerokość.}.

\section{Instrukcja użycia}
W ramach projektu jest zaimplementowany interfejs tekstowy (TUI).
Grę w trybie TUI otwieramy uruchamiając plik wykonywalny
halma-tui.py.
Po tym będziemy mieli opcję załadowania zapisu gry z pliku lub
utworzenia nowej. Gra może być uruchomiona w ustawieniu klasycznym
lub losowym. Mamy też możliwość wyboru graczy.
Po wyborze początkowych ustawień pojawi się plansza do gry.
\\~\\
Komunikacja z programem odbywa się przez okna dialogowe.
Możliwe do wyświetlenia okna dialogowe są wymienione
w pasku pomocy na górze okna. Żeby wyświetlić dane okno
dialogowe należy wcisnąć odpowiadający mu klawisz (q - quit,
m - move, s - save). Możemy też anulować operację wychodząc
z okna dialogowego klawiszem Escape.
\\~\\
Okien dialogowych są dwa rodzaje: Pytające o wybór opcji z listy i
pytające o inny ciąg znaków.
\\~\\
W oknie dialogowym pierwszego rodzaju należy wprowadzić w
polu tekstowym numer wybranej opcji i zatwierdzić
klawiszem enter.
\\~\\
Inaczej postępujemy z oknem dialogowym pytającym nas o ruch.
W nim należy wprowadzić ruch w formacie "`AB-CD"'\footnote{Wielkość liter nie ma znaczenia, nie należy natomiast wprowadzać spacji wewnątrz zapisu.}, gdzie AB
to współrzędne pola z krórego chcemy wykonać ruch, CD to
współrzędne pola na które chcemy wykonać ruch. Współrzędne
to para literek którymi podpisane są pola z boku planszy -
pierwsza współrzędna z lewej strony planszy, druga współrzędna
na górze.

\pagebreak
\section{Część refleksyjna}
Aplikacja ma przejrzystą architekturę z wyraźnie
oddzielonymi modułami. W zależnościach modułów nie
ma cykli. 
\\~\\
Interfejs użytkownika jest skalowalny i dostosowywuje się
do rozmiaru terminala. Sterowanie jest intuicyjne, a kolory
są dobrane precyzyjnie z dbałością o przejrzystość okna gry.
\\~\\
Są zaimplementowane cztery boty:
\begin{itemize}
	\item{RandomBot (wykonujący losowe ruchy)}
	\item{ForwardBot (ruszający się zawsze naprzód)}
	\item{MinimaxBot (wykorzystujący algorytm MiniMax)}
	\item{AgressiveMinimaxBot (wykorzystujący algorytm MiniMax)}
\end{itemize}

\noindent
Warto zwrócić uwagę
na różnicę między botami MinimaxBot i AgressiveMinimaxBot.
MinimaxBot minimalizuje różnicę odległości od przeciwległego rogu
pionków swoich i przeciwnika. Jego gra jest zachowawcza. Stara się
trzymać krawędzi planszy, nie podstawia kamieni tak żeby
przeciwnik mógł
je przeskoczyć. Niestety, nie potrafi doprowadzić swoich pionków do
obozu przeciwnika, woli ruszyć się do tyłu niż żeby przeciwnik
przeciwnik przeskoczył swoim pionkiem na drugi koniec planszy.
Próbą odpowiedzi na ten problem jest jego agresywna wersja -
AgressiveMinimaxBot.
\\~\\
Gra ma możliwość zapisu i wczytania z pliku. Istnieją duże możliwości
rozszerzenia
funkcjonalności dzięki podziałowi na moduły. Przykładowo, można
zaimplementować GUI które komunikowało by
się z resztą aplikacji przez taki sam interfejs jak już
napisany tekstowy interfejs. Można
zaimplementować więcej botów, a nawet możliwość gry zdalnej.
\\~\\
Jest nawet możliwość doimplementowania innej wersji TUI na
bazie interfejsu klasy TuiEngine.

\end{document}
