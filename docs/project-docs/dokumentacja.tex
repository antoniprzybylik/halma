\documentclass[12pt, titlepage]{article}
\usepackage[polish]{babel}
\usepackage[utf8]{inputenc}
\usepackage[T1]{fontenc}
\usepackage{nunito}
\usepackage[margin=1.2in]{geometry}
\usepackage{ragged2e}
\usepackage{graphicx}
\usepackage{grffile}
\usepackage{longtable}
\usepackage{wrapfig}
\usepackage{rotating}
\usepackage[normalem]{ulem}
\usepackage{amsmath}
\usepackage{textcomp}
\usepackage{amssymb}
\usepackage{capt-of}
\usepackage{hyperref}
\author{Antoni Przybylik\\ PIPR 22Z\\ GRUPA 101}
\date{\today}
\title{HALMA - Dokumentacja Projektu}

\setcounter{secnumdepth}{0}
\begin{document}
\maketitle
\justifying

\section{Zadanie}
Napisać program grający w Halma. Powinna być możliwość gry:

\begin{itemize}
\item dwóch osób ze sobą,
\item osoby z komputerem,
\item komputera z komputerem.
\end{itemize}

\noindent
Program powinien kontrolować poprawność wykonywanych ruchów. Interfejs z użytkownikiem może być tekstowy.
\\~\\
Bardzo istotną częścią zadania jest opracowanie i zaimplementowanie jak najlepszego algorytmu grającego w grę.
\\~\\
Moduł realizujący algorytm gry komputera musi być wydzielony.

\section{Cel i opis projektu}
Ante metus dictum at tempor. Placerat orci nulla pellentesque dignissim enim sit. Integer enim neque volutpat ac tincidunt vitae semper quis. Massa tempor nec feugiat nisl pretium fusce id velit ut. At elementum eu facilisis sed. Ac orci phasellus egestas tellus rutrum tellus. Nunc scelerisque viverra mauris in aliquam sem. Risus nullam eget felis eget nunc lobortis mattis aliquam. Sed euismod nisi porta lorem mollis. Amet commodo nulla facilisi nullam vehicula. Mauris nunc congue nisi vitae. Sit amet nisl purus in mollis nunc sed. Pellentesque pulvinar pellentesque habitant morbi. Ultricies integer quis auctor elit sed. Lobortis scelerisque fermentum dui faucibus in ornare quam. Tristique nulla aliquet enim tortor at auctor urna nunc.

\section{Architektura}
Opis architektury rozwiązania: podział programu na moduły z krótkim opisem. W opisie modułu powinien znaleźć się opis głównych klas (chodzi tu o krótki opis każdej z klas, nie należy opisywać każdej z metod/pól. Zdecydowanie nie należy wrzucać kawałków kodu).

\pagebreak
\section{Wymagania}
Żeby móc uruchomić grę w trbie TUI należy mieć zainstalowaną
bibliotekę curses.
\\~\\
Interfejs TUI działa tylko w terminalach obsługujących 8-bitowe
kolory i pozwalających na zmianę ich wartości\footnote{Każdy nowoczesny terminal jak urxvt, xfce4-terminal, gnome-terminal powinien spełniać te wymagania. Gra nie działa w xtermie.}.
Dodatkowo, jest wymagane żeby okno terminala miało rozmiar co
najmniej 40x70\footnote{Wysokość x Szerokość.}.

\section{Instrukcja użycia}
W ramach projektu jest zaimplementowany interfejs tekstowy (TUI).
Grę w trybie TUI otwieramy uruchamiając plik wykonywalny
halma-tui.py.
\\~\\
Komunikacja z programem odbywa się przez dialog boxy.
Są ich dwa rodzaje: Pytające o wybór opcji z listy i
pytające o inny ciąg znaków.
\\~\\
W dialog box'ie pierwszego rodzaju należy wprowadzić w
polu tekstowym numer wybranej opcji i zatwierdzić
klawiszem enter.
\\~\\
Inaczej postępujemy z dialog box'em pytającym nas o ruch.
W nim należy wprowadzić ruch w formacie "`AB-CD"'\footnote{Wielkość liter nie ma znaczenia, nie należy natomiast wprowadzać spacji wewnątrz zapisu.}, gdzie AB
to współrzędne pola z krórego chcemy wykonać ruch, CD to
współrzędne pola na które chcemy wykonać ruch. Współrzędne
to para literek którymi podpisane są pola z boku planszy.

\section{Część refleksyjna}
Część refleksyjna - tutaj powinniście Państwo podsumować zakres wykonanych prac oraz opisać rzeczy, których nie udało się osiągnąć. Całość proszę opatrzyć komentarzem - dlaczego coś nie został wykonane, na jakie nieprzewidziane przeszkody natrafiliście Państwo, a także co się zmieniło w stosunku do planowanego rozwiązania. W tym miejscu należałoby się też pochwalić, dlaczego Państwa projekt jest dobry i należy mu przyznać wysoką ocenę

\end{document}
