\documentclass[12pt, titlepage]{article}
\usepackage[polish]{babel}
\usepackage[utf8]{inputenc}
\usepackage[T1]{fontenc}
\usepackage{nunito}
\usepackage[margin=1.2in]{geometry}
\usepackage{ragged2e}
\usepackage{graphicx}
\usepackage{grffile}
\usepackage{longtable}
\usepackage{wrapfig}
\usepackage{rotating}
\usepackage[normalem]{ulem}
\usepackage{amsmath}
\usepackage{textcomp}
\usepackage{amssymb}
\usepackage{capt-of}
\usepackage{hyperref}
\author{Antoni Przybylik\\ PIPR 22Z\\ GRUPA 101}
\date{\today}
\title{HALMA - Dokumentacja Projektu}

\setcounter{secnumdepth}{0}
\begin{document}
\maketitle
\justifying

\section{Zadanie}
Napisać program grający w Halma. Powinna być możliwość gry:

\begin{itemize}
\item dwóch osób ze sobą,
\item osoby z komputerem,
\item komputera z komputerem.
\end{itemize}

\noindent
Program powinien kontrolować poprawność wykonywanych ruchów. Interfejs z użytkownikiem może być tekstowy.
\\~\\
Bardzo istotną częścią zadania jest opracowanie i zaimplementowanie jak najlepszego algorytmu grającego w grę.
\\~\\
Moduł realizujący algorytm gry komputera musi być wydzielony.

\section{Cel i opis projektu}
Celem projektu było napisanie gry Halma w Pythonie,
przygotowanie testów jednostkowych i dokumentacji.
W tym celu należało wykorzystać umiejętności zdobyte
podczas laboratoriów z przedmiotu PIPR.

\section{Architektura}
Projekt jest podzielone na trzy główne moduły:
\begin{itemize}
\item Silnik gry.
\item Interfejs użytkownika.
\item Klasy graczy.
\end{itemize}

\noindent
Dodatkowo zawiera jeszcze trzy klasy pomocnicze:
\begin{itemize}
\item Silnik UI.
\item Interfejs gry dla użytkownika.
\item Klasę Game reprezentującą stan całej gry.
\end{itemize}

\noindent
W klasie Engine zaimplementowana jest bazowa
funkcjonalność programu, między innymi metody
odpowedzialne za ustawianie/odczyt pól na planszy,
metody pozwalające na sprawdzenie możliwych ruchów
z danego pola. W tej klasie jest zapisany stan planszy
i numer obecnego ruchu.
\\~\\
W klasie HalmaTui zaimplementowano tekstowy interfejs
użytkownika dla gry Halma. Używa on surowych metod do
rysowania okienek i tak dalej, z klasy TuiEngine.
\\~\\
Klasy graczy (dziedziczące po Player) implementują metodę
make\_move(), która odpowiada za wykonywanie ruchu. W
przypadku botów ruch jest po prostu wykonywany. Gracze-ludzie
(TuiPlayer dziedziczący po Player) wyświetlają dialog box'a
przy użyciu interfejsu klasu TuiEngine (tej samej której używa
HalmaTui) pytającego gracza o ruch, a potem go wykonują.
\\~\\
Silnik UI znajdujący się w klasie TuiEngine implementuje metody
takie jak: rysująca dialog box'a, wyświetlająca splash screen.
\\~\\
Interfejs gry dla użytkownika znajduje się w klasie GameInterface.
Celem istnienia tej klasy jest wprowadzenie dodatkowej warstwy
abstrakcji pomiędzy UI, a silnik gry. Na przykład ruchy wpisane
przez użytkownika nie są sprawdzane i przetwarzane na współrzędne
w tych samych funkcjach które wyświetlają UI tylko są przekazywane
klasie GameInterface, która ma referencję na obiekt klasy Engine i wykonuje
ona ruch. Referencję na obiekt klasy GameInterface ma UI i TuiPlayer,
który jest obiektem klasy Player implementującym grę przez użytkownika.
\\~\\
Klasa Game posiada referencję na obiekty klas: GameInterface, Engine i
obiekty reprezentujące gracza białego i czarnego (obiekty klas
dziedziczących po Player). Ten stan można zapisać do pliku i wczytać
go z pliku.

\pagebreak
\section{Wymagania}
Żeby móc uruchomić grę w trbie TUI należy mieć zainstalowaną
bibliotekę curses.
\\~\\
Interfejs TUI działa tylko w terminalach obsługujących 8-bitowe
kolory i pozwalających na zmianę ich wartości\footnote{Każdy nowoczesny terminal jak urxvt, xfce4-terminal, gnome-terminal powinien spełniać te wymagania. Gra nie działa w xtermie.}.
Dodatkowo, jest wymagane żeby okno terminala miało rozmiar co
najmniej 40x70\footnote{Wysokość x Szerokość.}.

\section{Instrukcja użycia}
W ramach projektu jest zaimplementowany interfejs tekstowy (TUI).
Grę w trybie TUI otwieramy uruchamiając plik wykonywalny
halma-tui.py.
\\~\\
Komunikacja z programem odbywa się przez dialog boxy.
Są ich dwa rodzaje: Pytające o wybór opcji z listy i
pytające o inny ciąg znaków.
\\~\\
W dialog box'ie pierwszego rodzaju należy wprowadzić w
polu tekstowym numer wybranej opcji i zatwierdzić
klawiszem enter.
\\~\\
Inaczej postępujemy z dialog box'em pytającym nas o ruch.
W nim należy wprowadzić ruch w formacie "`AB-CD"'\footnote{Wielkość liter nie ma znaczenia, nie należy natomiast wprowadzać spacji wewnątrz zapisu.}, gdzie AB
to współrzędne pola z krórego chcemy wykonać ruch, CD to
współrzędne pola na które chcemy wykonać ruch. Współrzędne
to para literek którymi podpisane są pola z boku planszy.

\section{Część refleksyjna}
Aplikacja ma przejrzystą architekturę z modułami
oddzielonymi w sposób przejrzysty bez zagnieżdżania
i wstecznych referencji (topologia aplikacji jest
acyklicznym grafem skierowanym).
\\~\\
W grze jest możliwość zapisu i wczytania gry, są zaimplementowane
dwa boty - losowy i sprytny. Jest zaimplementowane wiele testów.
Interfejs użytkownika jest intuicyjny i łatwy w obsłudze, a
aplikacja posiada duże możliwości rozszerzenia dzięki enkapsulacji.
\\~\\
Przykładowo, można zaimplementować GUI które komunikowało by
się z resztą aplikacji przez taki sam interfejs jak TUI m. Można
zaimplementować więcej botów, a nawet możliwość gry zdalnej.
\\~\\
Jest nawet możliwość doimplementowania innej wersji TUI na
bazie interfejsu klasy TuiEngine.

\end{document}
